%=========================================================
\chapter{Introducción}


\cdtInstrucciones{
	Presentar el documento, indicando su contenido, a quien va dirigido, quien lo realizó, por que razón, dónde y cuando. \\
}
	Este documento contiene el Plan de trabajo del proyecto ``{\em Nombre del proyecto}'' correspondiente al trabajo realizado en el semestre 2016-2017-2 para la materia de Análisis y diseño orientado a objetos en el grupo 2CV9 por el equipo {\em Nombre del equipo}.

%---------------------------------------------------------
\section{Presentación}


\cdtInstrucciones{
	Indique el propósito del documento y las distintas formas en que puede ser utilizado.\\
}
	Este documento contiene el plan de trabajo detallando el alcance, tiempo, costo e involucrados. Tiene como objetivo dirigir el trabajo del proyecto y organizar al equipo de trabajo. Este documento debe ser aprobado por los principales responsables del proyecto.
	
	Este documento es el C1-PP1 del proyecto ``{\em Nombre del proyecto}''.
	
%---------------------------------------------------------
\section{Organización del contenido}

	En el capítulo \ref{cap:analisis} se presenta el análisis de la problemática identificada en la organización que sustenta el proyecto y es la base para la toma de desiciones a lo largo del mismo.
	
	En el capítulo \ref{cap:alcance} se presenta el objetivo y el alcance del proyecto, delineando los principales requerimientos funcionales y no funcionales del proyecto.
	
	En el capítulo \ref{cap:tiempo} se desglosan las actividades a realizar, el esfuerzo requerido y el calendario de actividades con base en la metodología o metodologías elegidas para este proyecto.
	
	En el capítulo \ref{cap:capHumano} ...
	
	En el capítulo \ref{cap:comunicacion} ...

%---------------------------------------------------------
\section{Notación, símbolos y convenciones utilizadas}

	En este documento se utiliza un diagrama de Gantt para presentar el calendario de actividades.
	
	Los requerimientos del sistema se enumeran utilizando la notación RS1, RS2, RS3, etc.
	
	Se utilizan letras en {\em cursivas} para indicar palabras de otro idioma o que requieren una atención específica. 
	
	La mayoría de las aclaraciones sobre un elemento se colocan como notas al pie.

