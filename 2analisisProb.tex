%=========================================================
\chapter{Análisis del problema}
\label{cap:analisis}

Este capítulo contiene la justificación del proyecto. Presenta una breve descripción de la empresa, un análisis del problema identificando sus principales causas y termina con la estimación de las consecuencias más importantes a mediano y largo plazo. Para este análisis se utilizaron las técnicas de ``lluvia de ideas'', ``Diagrama de ishikahua'' y ``Análisis FODA''.

%---------------------------------------------------------
\section{Descripción del contexto}

\cdtInstrucciones{Realizar una descripción de la situación de la empresa, describiendo sus áreas, organización, procesos, y todos los factores que se considere relevante conocer para el análisis del problema y el desarrollo del sistema propuesto.}

%---------------------------------------------------------
\section{Descomposición del problema}

\cdtInstrucciones{
	Presentar el problema general y la lista de los problemas desglosados, relaciones, organizados y priorizados para su análisis.\\ 
	Cerrar la sección con una apreciación o conclusión.
}

%---------------------------------------------------------
\section{Análisis de causas y consecuencias}

\cdtInstrucciones{
	Se presenta un análisis que describe el procesos y lista las causas probables de cada problema identificado, argumentando una justificación que la delata como el origen de la problemática.\\
	Considere que en ocasiones, una causa solo es un factor y se convierte en causa de un problema cuando está combinada con otros factores.
}

%---------------------------------------------------------
\section{Síntesis y propuesta de solución}

\cdtInstrucciones{
	Haga una investigación sobre soluciones existentes, técnicas o herramientas aplicables al problema, discuta sobre sus ventajas y desventajas e integre una o mas alternativas de solución bosquejando sus características y beneficios.
}

% - - - - - - - - - - - - - - - - - - - - - - - - - - - - - 
\subsection{Estado del arte}

\cdtInstrucciones{
	Presentación del resultado de la investigación de alternativas de solución, soluciones existentes que pueden ser aplicables e ideas innovadoras sobre la problemática identificada. \\
	
	En esta sección se presentan las soluciones encontradas o ideas principales que se desea aplicar para cada uno de los problemas independientes o en grupos según se considere necesario.\\
	
	No olvide argumentar la razón por la cual es importante cada solución, de que manera resuelve la problemática, así como sus ventajas y desventajas.
}

% - - - - - - - - - - - - - - - - - - - - - - - - - - - - - 
\subsection{Alternativas de solución}

\cdtInstrucciones{
	Una vez presentadas las soluciones independientes integre al menos una solución integral como estrategia base del proyecto. Explique para cada solución su importancia, la forma en que resuelve la problemática, sus ventajas y desventajas y sus limitantes, es decir, en caso de que no resuelva al 100\% la problemática resalte que parte de los problemas identificados o descritos no se resolverán por la solución planteada.
}

% - - - - - - - - - - - - - - - - - - - - - - - - - - - - - 
\subsection{Propuesta seleccionada}

\cdtInstrucciones{
	Indique, en caso de que hayan más de una solución en la sección anterior, la solución seleccionada para ser la base del presente proyecto.
}

